\documentclass{beamer}
\usefonttheme{professionalfonts}
\mode<presentation>
{
	\usetheme{Warsaw}
	\setbeamercovered{transparent}
}
\usepackage{amsmath}
\usepackage{amssymb}
\usepackage{listings}
\usepackage{graphicx}
\lstset{escapechar=@,
	basicstyle=\ttfamily\color{black},
	keywordstyle=\color{blue}\bfseries,
	identifierstyle=\color{red},
	commentstyle=\color{darkgray},
	stringstyle=\ttfamily,
	showstringspaces=false,
}
\usefonttheme{serif}
\title[The Saga Begins]{A Bitter Old Man's Haskell Guide \\ \small{The Saga Begins}}
\author[VALEFOR, VARIK]{VALEFOR, VARIK \\ $<$varikvalefor@aol.com$>$}
\begin{document}
	\maketitle
	\tableofcontents
	\begin{frame}{About the Speaker}
		VARIK VALEFOR
			\begin{itemize}
				\item Intelligence analyst
				\begin{itemize}
					\item \textbf{\textit{Not}} a computer programmer!!!
					\item Can only answer a few questions re: work
				\end{itemize}
				\item Diverse set of skills
			\begin{itemize}
				\item Mathematics!
				\begin{itemize}
					\item Can't add numbers or do other low-level things
					\item Dove into calculus in the sixth grade
					\item Concepts understood and applied well
					\item Computers made for a reason
				\end{itemize}
				\item Computer programming
				\begin{itemize}
					\item Generally not enjoyed
					\item Haskell is a favourite language
					\item Obsoletes low-to-mid-level mathematical tasks
				\end{itemize}
				\item Motivating similar people via shouting
			\end{itemize}
		\end{itemize}
	\end{frame}
	\section{What is Haskell?}
		\subsection{In a Nutshell}
			\begin{frame}{In a Nutshell}
				Haskell is\dots
				\begin{itemize}
					\item A functional language
					\item TURING-complete
					\item \textit{\textbf{Applicable in the real world!}}
				\end{itemize}
			\end{frame}
		\subsection{What Haskell Lacks}
			\begin{frame}{What Haskell Lacks}
				Haskell\dots
				\begin{itemize}
					\item Has no ``traditional'' loops
					\begin{itemize}
						\item Ex. \texttt{for(int i=0;i<90210;i++) printf("\%i ", i);}
						\item \texttt{map} often used instead\\
							\texttt{mapM ($\backslash$a -> putStr \$ show a ++ " ") [0..90210];}
						\item Recursion also used
					\end{itemize}
					\item \dots's Non-IO functions cannot perform IO operations
					\begin{itemize}
						\item \texttt{int b(int c) \{printf('ass'); return 1;\}}
							has no Haskell equivalent
						\item Instead, monadic functions are used
						\begin{itemize}
							\item Oh\dots monads.
						\end{itemize}
					\end{itemize}
				\end{itemize}
			\end{frame}
		\subsection{What Haskell \textit{Has}}
			\begin{frame}{What Haskell \textit{Has}}
				Haskell has\dots
				\begin{itemize}
					\item A formal specification
					\begin{itemize}
						\item All aspects of language outlined
						\item Freely available and modifiable
					\end{itemize}
					\item Monads
					\begin{itemize}
						\item Sometimes scare the piss out of beginners
						\item ``Very unique''
					\end{itemize}
					\item Strong typing
					\item Lazy execution
					\begin{itemize}
						\item Bad name; good thing
						\item ``Stuff calculated only if stuff actually used''
					\end{itemize}
					\item The Glasgow Haskell Compiler
					\begin{itemize}
						\item Fast; speeds rival the speeds of
							good C programs
						\item REPL via GHCi
					\end{itemize}
				\end{itemize}
			\end{frame}
			\section{\textit{Main} Advantages of Using Haskell}
		\subsection{Functional Nature}
			\begin{frame}{The Advantages of Functional Programming}
				\begin{itemize}
					\item Stronger resemblance to mathematical notation
					\item Understanding \textit{well-written} programs easy
					\item Functions' purposes can be approximately determined
						by inspecting functions' types
					\item Solving certain problems becomes trivial
				\end{itemize}
			\end{frame}
		\subsection{High Maintainability}
			\begin{frame}{Maintainability}
				\begin{itemize}
					\item Terse, high-level
					\begin{itemize}
						\item Programs fairly easily re-written
						\item Understanding \textit{well-written} programs easy
					\end{itemize}
					\item Strong type system enforced
					\begin{itemize}
						\item ``This function must be of this type.''
						\item Simplifies writing solutions for small problems
					\end{itemize}
				\end{itemize}
			\end{frame}
		\subsection{Speed}
			\begin{frame}{Writing Haskell Programs}
				\begin{itemize}
					\item Reading and writing Haskell programs easy after language
						understood
					\item Compilers' output can be stupidly fast
					\begin{itemize}
						\item \dots but only when given good programs
					\end{itemize}
				\end{itemize}
			\end{frame}
	\section{Potential Disadvantages of Using Haskell}
		\subsection{Extremely Different}
			\begin{frame}{Declarative}
				Declarative programming\dots
				\begin{itemize}
					\item Extremely high-level; often a source of difficulty
					\begin{itemize}
						\item Many Stack Overflow questions re: Haskell
							\texttt{while} loops 
					\end{itemize}
					\item ``Computer is told to do $X$, instead of being given
						instructions for doing $X$''
					\item Requires approaching problems differently
					\begin{itemize}
						\item Case in point: no loops; no variable redefinitions
					\end{itemize}
				\end{itemize}
			\end{frame}
			\begin{frame}[fragile]{Example of Not Using Loops}
				\small
				\begin{lstlisting}[language=C++]
bool isPrime(bigint x)
{
	bool k = true;
	for(bigint j = 2; j < x/2; j++)
		if (x % j == 0)
			k = false;
	return k;
}
				\end{lstlisting}
				is equivalent to\dots
				\begin{lstlisting}[language=Haskell]
isPrime :: Integer -> Bool;
isPrime j = filter ((==0) . (mod j)) f == []
        where f = [2..b j]
              b = floor . sqrt . fromIntegral
              --Integer SQRT defined above
				\end{lstlisting}
			\end{frame}
			\begin{frame}{Monads}
				\begin{columns}
					\begin{column}{0.75\textwidth}
						Monads are\dots
						\begin{itemize}
							\item Integral to Haskell, with \texttt{>>=} even being
								visible in some logos
							\item Not terribly easily explained
							\item Likely the real challenge of this video series
							\item Like burritos..?
							\item ``Wrappers'' for things..?
							\item Really a blessing in disguise
						\end{itemize}
					\end{column}
					\begin{column}{0.25\textwidth}
						\includegraphics[width=1\textwidth]{x-mow-nads.png}
					\end{column}
				\end{columns}
			\end{frame}
			\begin{frame}{Relatively Strange Syntax}
				\begin{itemize}
					\item \$ operator
					\item \texttt{>>=}
					\item Relatively akin to mathematical notation
					\item Parentheses not used by functions
					\item Semicolons unnecessary
				\end{itemize}
			\end{frame}
		\subsection{Relatively Uncommon}
			\begin{frame}{Relatively Uncommon}
				Haskell is relatively uncommon.
				\begin{itemize}
					\item Less assistance available
					\item Fewer compilers available
					\item Fewer libraries available
					\item \texttt{putStrLn \$ "So on" ++ cycle ", and so on" ++ "."}
				\end{itemize}
			\end{frame}
	\section{Onto the Point\dots}
		\subsection{Why Learn Haskell?}
			\begin{frame}{Why Learn Haskell?}
				\begin{itemize}
					\item Forces using different solutions
					\begin{itemize}
						\item Haskell's solutions differ from imperative solutions
					\end{itemize}
					\item Haskell is actually a very practical language
					\begin{itemize}
						\item Haskell can compute all computable things
						\item Solutions created quickly
						\item Such solutions are fast \textit{and maintainable}
						\item Source code fully portable
					\end{itemize}
				\end{itemize}
			\end{frame}
		\subsection{Why Use this Guide?}
			\begin{frame}{Why Use this Guide?}
				This video series should introduce the viewer to Haskell such
				that this introduction is relatively\dots
				\begin{itemize}
					\item Humorous
					\item Enjoyable
					\item Focused on practical stuff
					\item Hopefully a bit relaxing, too
					\item Most future videos primarily feature the text editor/CLI
				\end{itemize}
				Think of a video equivalent of
				\textit{Learn You a Haskell for Great Good}.
			\end{frame}
		\subsection{If You're Still Here\dots}
			\begin{frame}{If You're Still Here, then\dots}
				\huge{Let's go!}\footnote{\Tiny{The next video might not be
						released for a little while.  Whoops.}}
			\end{frame}
\end{document}
%We started from the bottom; now, we're... still at the bottom.  Shit.
